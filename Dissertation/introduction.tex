\chapter*{Введение}                         % Заголовок
\addcontentsline{toc}{chapter}{Введение}    % Добавляем его в оглавление

В современном мире данные все чаще представляются и хранятся в виде графов, которые часто используются в специализированных базах данных. Для извлечения информации из таких графов обычно применяются строковые языки запросов, однако данный подход имеет свои недостатки. В данной работе представлен подход к поиску путей в графе с ограничениями, основанный на использовании парсер-комбинаторов.

Целью данной работы является разработка библиотеки на языке Kotlin, которая позволит задавать ограничения на поиск путей в графе путем описания грамматики через комбинаторы. Реализация библиотеки включает в себя различные алгоритмы, обеспечивающие оптимальную скорость работы, поддержку недетерминированных и леворекурсивных грамматик, а также обнаружение циклов в графе. Предложенный подход решает ряд проблем, существующих у аналогов с парсер-комбинаторами.

В данной работе проведен сравнительный анализ времени работы разработанного решения и существующей библиотеки Meerkat, которая является наиболее быстрой библиотекой парсер-комбинаторов на данный момент. 