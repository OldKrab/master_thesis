\chapter*{Введение}                         % Заголовок
\addcontentsline{toc}{chapter}{Введение}    % Добавляем его в оглавление

В современном мире многие данные представляются и хранятся в виде графов, которые часто используются в специализированных базах данных. 
Чтобы извлечь нужные нам данные, мы можем получить все пути из графа, которые удовлетворяют определенным ограничениям. 
Описать запрос с такими ограничениями можно, например, с помощью формальной грамматики. 

Обычно для извлечения информации из таких графов применяются строковые языки запросов, такие как SPARQL~\cite{sparql}, Cypher~\footnote{Веб-сайт Neo4j Cypher: https://neo4j.com/docs/cypher-manual (дата обр. 21.05.2024)} и другие. Но данные языки поддерживают только регулярные грамматики.

В свою очередь, существуют примеры использования контекстно-свободных (КС) грамматик для задания ограничений на пути в графе. Например, множество проблем статического анализа программ можно свести к поиску путей в графе с контекстно-свободными ограничениями \cite{RepsProgramAnalysis}.
Также КС-ограничения применяются для графов в области биоинформатики \cite{BioinformaticsCF}, для запросов к RDF-графам \cite{RDF_CF} и т.д.

При разработке приложения мы хотим иметь возможность работать с данными, используя тот же язык общего назначения. 
Это можно делать, например, используя строковые языки, описанные ранее. Такой подход имеет ряд недостатков, такие как отсутствие проверок корректности во время компиляции или отсутствие поддержки в IDE. Другим подходом может быть использование ORM (Object-Relational Mapping), но с ним тяжело декомпозировать и переиспользовать части запроса.

В данной работе предлагается использовать парсер-комбинаторы~\cite{Hutton_1992} для описания ограничений. Данный подход позволяет описывать грамматику для путей в графе, используя язык общего назначения. Использование парсер-комбинаторов для запросов к графам уже было представлено в работах \cite{Trails} и \cite{MeerkatGraphs}, но каждое из этих решений имеет ряд недостатков. 

Целью данной работы является разработка библиотеки, которая позволит задавать ограничения на поиск путей в графе путем описания грамматики через комбинаторы. Реализация библиотеки включает в себя различные алгоритмы, обеспечивающие оптимальную скорость работы, поддержку контекстно-свободных грамматик, в том числе недетерминированных и леворекурсивных, а также обнаружение циклов в графе. Предложенный подход решает ряд проблем, существующих у аналогов с парсер-комбинаторами.

В данной работе также проведен сравнительный анализ времени работы разработанного решения и существующей библиотеки Meerkat~\cite{MeerkatGraphs}, которая является наиболее быстрой библиотекой парсер-комбинаторов на данный момент. 