\chapter{Экспериментальное тестирование}
\label{ch:ch4}


Для анализа полученного результата был проведено сравнение времени работы с библиотекой Meerkat. В свою очередь сравнение библиотеки Meerkat c библиотекой Trails приведено в работе \cite{MeerkatGraphs}. Замер производился с помощью фреймворка JMH\footnote{https://github.com/openjdk/jmh (Дата обр. 21.05.2024)}.

Тестирование проводилось на вычислительной машине с 12-ядерным процессором AMD Ryzen 9 7900X и 128 ГБ оперативной памяти DDR5.

Для запуска тестов была добавлена поддержка Neo4j графов, т.к. эту базу данных поддерживает Meerkat. Для этого использовалось предоставляемое Neo4j Java API\footnote{https://neo4j.com/docs/java-reference/4.4/ (дата обр. 21.05.2024)}, с помощью которого был реализован интерфейс графа, с которым работают комбинаторы.

Первый набор тестов был взят такой же, какой использовали авторы расширения Meerkat, в свою очередь он был представлен в работе \cite{RDF_CF}. Это набор онтологий, каждая из которых представлена RDF-файлом. Для тестирования эти файлы были преобразованы в ориентированные графы, каждый из которых был загружен в свою базу данных Neo4j.

\begin{figure}[htp]
    \centering
    \begin{subfigure}[b]{0.49\textwidth}
        \centering
        $Q_1 \to subclassof^{-1}\ Q1?\ subclassof$\\
        $Q_1 \to type^{-1}\ Q1?\ type$\\
        \caption{Запрос 1}
    \end{subfigure}
    \hfill
    \begin{subfigure}[b]{0.49\textwidth}
        \centering
        $S \to subclassof^{-1}\ S?\ subclassof$
        $Q_2 \to S\ subclassof$
        \caption{Запрос 2}
    \end{subfigure}
    \caption{Запросы для тестирования}
    \label{fig:queries}
\end{figure}

Используемые запросы приведены на рисунке \ref{fig:queries}. Ключевые слова, например, \texttt{type} или \texttt{subclassof}, являются метками на ребрах. Верхний индекс \texttt{-1} обозначает обратное направление ребра. Знак \texttt{?} обозначает ноль или одно применение парсера.

\begin{lstlisting}[float=h, label=lst:qqueries, caption={Запросы, используя комбинаторы}]
val subclassof1 = throughInE { it.label == "subClassOf" }
val subclassof = throughOutE { it.label == "subClassOf" }
val type1 = throughInE { it.label == "type" }
val type = throughOutE { it.label == "type" }
val Q1 = LazyParser<Neo4jVertexState, Neo4jVertexState, Any>()
Q1.p = (subclassof1 seq (Q1 or eps()) seq subclassof) or
        (type1 seq (Q1 or eps()) seq type)

val S = LazyParser<Neo4jVertexState, Neo4jVertexState, Any>()
S.p = (subclassof1 seq (S or eps()) seq subclassof)
val Q2 = ((S or eps()) seq subclassof)
\end{lstlisting}

Описание этих запросов, используя комбинаторы, приведено в листинге \ref{lst:qqueries}. Комбинаторы \texttt{throughOutE} и \texttt{throughInE} вспомогательные: они пропускают выходящую или входящую вершину ребра соответственно.

\begin{table}[h]
    \caption{Информация о RDF графах}
    \begin{tabular}{|l|c|c|c|}
        \hline
        \textbf{Граф} & \textbf{Кол-во вершин} & \textbf{Кол-во рёбер} \\
        \hline
        atom-primitive.owl & 291 & 685 \\
        biomedical-measure-primitive.owl & 341 & 711 \\
        foaf.rdf & 256 & 815 \\
        generations.owl & 129 & 351 \\
        people\_pets.rdf & 337 & 834 \\
        pizza.owl & 671 & 2604 \\
        skos.rdf & 144 & 323 \\
        travel.owl & 131 & 397 \\
        univ-bench.owl & 179 & 413 \\
        wine.rdf & 733 & 2450 \\
        \hline
\end{tabular}
\label{tab:rdfGraphs}
\end{table}

Характеристики графов-онтологий приведены в таблице \ref{tab:rdfGraphs}. Результаты  замеров приведены в таблице \ref{tab:rdfTimes}. 

\begin{table}[h]
    \caption{Время выполнения для моего решения и Meerkat}
    \begin{tabular}{|b{0.36\linewidth}|b{0.13\linewidth}|b{0.12\linewidth}|b{0.13\linewidth}|b{0.12\linewidth}|}
        \hline
        \textbf{Граф} & \multicolumn{2}{|c|}{\textbf{Запрос 1 (мс)}} & \multicolumn{2}{|c|}{\textbf{Запрос 2 (мс)}} \\
        \cline{2-5}
         & \textbf{Мое решение} & \textbf{Meerkat} & \textbf{Мое решение} & \textbf{Meerkat} \\ 
        \hline
        atom-primitive.owl  & 19.498 & 22.013 & 11.378 & 12.593 \\
        biomedical-mesure-primitive  & 23.537 & 42.514 & 6.464 & 8.435 \\
        foaf.rdf  & 3.337 & 3.485 & 0.398 & 0.432 \\
        generations.owl  & 1.273 & 1.432 & 0.167 & 0.166 \\
        people\_pets.rdf  & 8.334 & 12.692 & 0.581 & 0.676 \\
        pizza.owl  & 61.411 & 123.946 & 9.605 & 9.595 \\
        skos.rdf  & 0.709 & 0.724 & 0.179 & 0.185 \\
        travel.owl  & 2.822 & 4.834 & 0.377 & 0.406 \\
        univ-bench.owl  & 2.889 & 3.633 & 0.467 & 0.482 \\
        wine.rdf  & 55.053 & 146.174 & 1.452 & 1.694 \\
        \hline
    \end{tabular}
    \label{tab:rdfTimes}
\end{table}

Видно, что почти на всех графах мое решение выигрывает по времени выполнения. 

Данные онтологии имеют относительно небольшой объем, поэтому было проведено тестирование на более крупных графах. В качестве таковых были взяты некоторые графы из набора CFPQ\_Data\footnote{https://formallanguageconstrainedpathquerying.github.io/CFPQ\_Data/graphs (дата обр. 21.05.2024)}, у которых количество вершин и ребер на порядки больше, чем было в онтологиях.

\begin{table}[h]
    \caption{Информация о CFPQ графах и время выполнения}
    \begin{tabular}{|l|c|c|c|c|c|}
        
    \hline
    \textbf{Граф} & \textbf{Кол-во вершин} & \textbf{Кол-во рёбер} & \multicolumn{2}{|c|}{\textbf{Запрос 1 (с)}} \\
    \cline{4-5}
    & & & \textbf{GParsers} & \textbf{Meerkat} \\
    \hline
    eclass.csv & 239111 & 360248 & 1.220 & 1.095 \\
    enzyme.csv & 48815 & 86543 & 0.121 & 0.116 \\
    geospecies.csv & 450609 & 2201532 & 1.755 & 1.471 \\
    go.csv & 582929 & 1437437 & 6.231 & 5.367 \\
    go\_hierarchy.csv & 45007 & 490109 & 12.015 & 8.337 \\
    \hline
    \end{tabular}
    \label{tab:CFPQTimes}
\end{table}

Характеристики этих графов и время выполнения моего решения и Meerkat можно увидеть в таблице \ref{tab:CFPQTimes}. Видно, что мое решение несколько хуже решения Meerkat, но в целом мы получаем сопоставимое время работы.

Еще один замер был проведен на наборе данных Yago\footnote{https://gitlab.inria.fr/tyrex-public/rlqdag (дата обр. 21.05.2024)}. Данный граф имеет 62,643,951 ребер и 42,832,856 вершин. В качестве запроса был взят первый из предложенного списка, который выглядит так на языке Cypher:
\begin{Verbatim}
MATCH (x)-[:P74636308]->()-[:P59561600]->
    ()-[:P13305537*1..]->()-[:P92580544*1..]->
    (:Entity{id:'40324616'}) RETURN DISTINCT x
\end{Verbatim}

В этом запросе используется как метки на ребрах, так и свойство вершины \texttt{id}. 

Написав данный запрос на комбинаторах в том же виде, не удалось выполнить запрос. Алгоритм перебирает все вершины, и на каждой происходит различная мемоизация, из-за чего не хватает памяти вычислительной машины. Было решено переписать этот запрос в обратную сторону, выполнив поиск по обратным ребрам, тогда получится сразу отфильтровать все вершины, которые не имеют свойство \texttt{id} равное \texttt{'40324616'}. 

Полученный в итоге запрос на комбинаторах в моем решении выглядит так:
\begin{lstlisting}
(v { it.properties["id"] == "40324616" } seq
(inE { it.label == "P92580544" } seq inV()).some seq
(inE { it.label == "P13305537" } seq inV()).some seq
inE { it.label == "P59561600" } seq inV() seq
inE { it.label == "P74636308" } seq inV())
\end{lstlisting}

А запрос, написанных на комбинаторах Meerkat выглядит так:
\begin{lstlisting}
syn(V((e: Entity) => e.getProperty("id") == "40324616") ~
inE((e: Entity) => e.label() == "P92580544").+ ~
inE((e: Entity) => e.label() == "P13305537").+ ~
inE((e: Entity) => e.label() == "P59561600") ~
inE((e: Entity) => e.label() == "P74636308"))
\end{lstlisting}

В итоге мое решение выдало верный результат, аналогичный запросу на языке Cypher. Время выполнения моего решения составило 5.558 секунд. Решение Meerkat, к сожалению, не укладывается в аппаратные ограничения по памяти. Запрос на языке Cypher отрабатывает за 215 мс, что вполне ожидаемо, т.к. Cypher может использовать внутренние структуры данных neo4j, например индексы. 
