\chapter*{Synopsis}
\addcontentsline{toc}{chapter}{Synopsis} 

\begin{center}
    General thesis summary
\end{center}


\paragraph*{Relevance of the chosen topic.}
\paragraph*{Goal.}
\paragraph*{Objectives.}
\paragraph*{Research methods.}
\paragraph*{Assertions that are presented for defense.}
\paragraph*{The novelty of research.}
\paragraph*{The scientific and technical objective.}
\paragraph*{The research object.}
\paragraph*{The research subject.}
\paragraph*{The theoretical significance.}
\paragraph*{The practical significance.}
\paragraph*{The accuracy of the obtained results.}

\paragraph*{Implementation of research results.}
\paragraph*{Approbation of research results.}
\paragraph*{Personal contribution of the author.}
\paragraph*{Thesis structure and number of pages.}

Thesis consists of the introduction,
\formbytotal{totalchapter}{chapter}{}{s}{},
conclusion and 
\formbytotal{totalappendix}{appendix}{}{es}{}.
Thesis is 
\formbytotal{TotPages}{page}{}{s}{} long, including
\formbytotal{totalcount@figure}{figure}{}{s}{} and
\formbytotal{totalcount@table}{table}{}{s}{}.
Bibliography consists of
\formbytotal{citenum}{item}{}{s}{}.


\newpage
\section*{Main contents of the work}

In Chapter~\ref{ch:ch1}...

\section*{Publications.}

Key results of research are described in \theAllMyPapers~publications. 
Among them
%Four of them are published in journals recommended by the Higher Attestation Commission,
\theScopusPapers~is published in a journal indexed by Scopus. 
%One certificate of state registration of a computer program has also been obtained.



Publications in international journals indexed by Scopus:
\begin{refsection}[biblio/own.bib]
\nocite{*}
\printbibliography[
    keyword=scopus,
    %title={В международных изданиях, индексируемых в базе данных Scopus}, 
    %heading=subbibliography,
    heading=none,
    resetnumbers=true
]
\end{refsection}


List of all relevant author's publications:
\begin{refsection}[biblio/own.bib]
\nocite{*}
\printbibliography[
    keyword=own,
    %title={Список всех публикаций автора по теме диссертации}, 
    %heading=subbibliography,
    heading=none,
    resetnumbers=true
]
\end{refsection}