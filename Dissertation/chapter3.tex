\chapter{Реализация библиотеки парсер-комбинаторов}
\label{ch:ch3}

В качестве языка программирования для реализации библиотеки парсер-комбинаторов был взят Kotlin.
Он поддерживает функции высшего порядка, а также имеет удобный синтаксис для анонимных функций.
Помимо этого он поддерживает функции в инфиксной нотации, что будет удобно для комбинирования парсеров.


\section{Базовая структура парсера}

Изначально была разработана базовая архитектура библиотеки, комбинаторы в которой были реализованы через рекурсивный спуск.
Такие комбинаторы вызывали другие рекурсивно, что могло привести к бесконечному выполнению на графах с циклами или при наличии левой рекурсии в определении парсера.
Даже без таких особенностей мы могли получить переполнение стека вызовов на достаточно объемном входе.
Но данный подход обладает простотой реализации, что позволяет относительно быстро разработать базовую архитектуру и различные комбинаторы.

Рассмотрим тип парсера:
\begin{lstlisting}
BaseParser<I, O, R>(
    val f: (I) -> Sequence<ParserResult<O, R>>)
\end{lstlisting}
За основу был взят тип из библиотеки Trails. Парсер параметризуется тремя типами: типы входящего и выходящего состояния и тип результата.

Парсер по сути является функцией, которая принимает входящее состояние и возвращает последовательность выходящих состояний и результатов. В данном случае \verb|ParserResult| представляет из себя пару. Класс \verb|Sequence| --- это стандартная структура данных в Kotlin, которая представляет из себя ленивую последовательность, которая вычисляется в процессе извлечения из нее данных. Здесь \verb|Sequence| используется из-за того, что результатов может быть бесконечно много, например, при наличии циклов в графе, а \verb|Sequence| позволяет лениво извлекать эти результаты.

В Trails вместе с входящим состоянием передается окружение, которое как-то связанно с этим состоянием, например, это можем быть граф или строка. В предлагаемом решении это окружение содержится внутри самого состояния.

В контексте работы с графами было введено три различных типа состояний: стартовое состояние, состояние на ребре и состояние на вершине. Стартовое состояние обозначает, что парсер принимает весь граф, и может начать разбор с любой вершины или ребра. Состояние на ребре хранит конкретное ребро, с которого нужно продолжить разбор. Аналогично про состояние на вершине. Все эти состояния хранят ссылку на граф, с которым они ассоциированы.

Чтобы работать с различными реализациями графов, я написал интерфейс графа, которому эти реализации должны соответствовать. Этот интерфейс будут использовать базовые комбинаторы графа, описанные в следующем разделе.

\section{Парсер-комбинаторы}

В данном разделе рассмотрим реализацию парсер-комбинаторов. Условно их можно разделить на две группы: базовые и объединяющие комбинаторы. Первые используются, чтобы описать самые простые ограничения, например, на одно ребро или вершину. Вторые комбинаторы позволяют конструировать более сложные ограничения, объединяя другие комбинаторы.

\subsection{Базовые комбинаторы}

Рассмотрим базовые комбинаторы для разбора графа, из которых с помощью других комбинаторов будут конструироваться сложные запросы.

Чтобы создать парсер, который будет принимать стартовое состояние, используются комбинаторы \verb|v| и \verb|edge|:

\begin{nobreaks}
    \begin{lstlisting}
fun v(p: (V) -> Boolean = { true })
    : BaseParser<StartState<V, E>, VertexState<V, E>, V>
fun edge(p: (E) -> Boolean = { true })
    : BaseParser<StartState<V, E>, EdgeState<V, E>, E> 
\end{lstlisting}
\end{nobreaks}

Функция \verb|v| принимает предикат, по которому все множество вершин графа фильтруется, и каждая подходящая вершина возвращается как выходящее состояние и результат.
Аналогично функция \verb|edge|.
Для этих парсеров интерфейс графа содержит методы для получения последовательности всех вершин и ребер.

Для случая, когда мы получаем состояние на вершине и хотим разобрать входящие или исходящие ребра, мы можем воспользоваться следующими комбинаторами:

\begin{nobreaks}
\begin{lstlisting}
fun outE(p: (E) -> Boolean = { true })
    : BaseParser<VertexState<V, E>, EdgeState<V, E>, E>
fun inE(p: (E) -> Boolean = { true })
    : BaseParser<VertexState<V, E>, EdgeState<V, E>, E>
\end{lstlisting}
\end{nobreaks}

Функции \verb|outE| и \verb|inE| принимают предикат, по которому фильтруют множества исходящих или входящих ребер текущей вершины. Соответственно, возвращают такие парсеры состояния на ребрах и сами ребра как результаты.

Если мы находимся на ребре и хотим разобрать начальную или конечную вершину этого ребра, то можно использовать аналогичные комбинаторы \verb|outV| и \verb|inV|:
\begin{lstlisting}
fun outV(p: (V) -> Boolean = { true })
    : BaseParser<EdgeState<V, E>, VertexState<V, E>, V>
fun inV(p: (V) -> Boolean = { true })
    : BaseParser<EdgeState<V, E>, VertexState<V, E>, V>
\end{lstlisting}

Интерфейс графа, для поддержки комбинаторов вершин и ребер, должен позволять получать множество входящих или исходящий ребер для конкретной вершины, а также получать начальную или конечную вершину для конкретного ребра.

Также были добавлены более общие базовые комбинаторы: \verb|success|, \verb|eps| и \verb|fail|.
Комбинатор \verb|success| создает парсер, который возвращает переданный аргумент как результат разбора, не изменяя входящее состояние.
Комбинатор \verb|eps| --- частный случай \verb|success|, возвращающий \verb|Unit|\footnote{В других языках данный тип называют \texttt{void}.}.
Комбинатор \verb|fail| создает парсер, не возвращающий результатов и не изменяющий состояние.
Их сигнатуры следующие:
\begin{lstlisting}
fun success(v: R): BaseParser<S, S, R>  
fun eps(): BaseParser<S, S, Unit>
fun fail(): BaseParser<S, S, R>
\end{lstlisting}

Базовая структура парсера, описанная в предыдущем разделе, позволяет создавать не только парсеры для графов, но и для строк.
В рамках тестирования полученной архитектуры были разработаны комбинаторы для создания строковых парсеров.
Они позволяют создавать парсеры из строки или регулярного выражения.

\begin{lstlisting}[float=ht, label=lst:stringParser, caption=Пример строкового парсера]
val expr = LazyParser<StringPos, StringPos, Int>()
expr.p = rule(
    (expr seql "+".p seq expr) using { x, y -> x + y },
    (expr seql "−".p seq expr) using { x, y -> x - y },
    "(".p seqr expr seql ")".p,
    "[0−9]*".toRegex().p using { it.toInt() })
\end{lstlisting}

В листинге \ref{lst:stringParser} мы создаем парсер, способный разбирать и вычислять арифметические выражения с операциями \verb|+|, \verb|-| и группирующими скобками.


\subsection{Объединяющие комбинаторы}

Рассмотрим реализованные в рамках этой работы комбинаторы для объединения парсеров.

\begin{lstlisting}[float=ht, label=lst:combinators, caption=Объединяющие комбинаторы]
fun BaseParser<I, O1, R1>.seq(p2: BaseParser<O1, O2, R2>)
                : BaseParser<I, O2, Pair<R1, R2>>

fun BaseParser<I, O, R>.or(p2: BaseParser<I, O, R>)
                : BaseParser<I, O, R>

fun rule(first: BaseParser<I, O, R>, 
        vararg rest: BaseParser<I, O, R>)

fun BaseParser<I, O, A>.using(f: (A) -> B): BaseParser<I, O, B>

fun BaseParser<I, O, R>.that(constraint: BaseParser<O, O2, R2>)
                : BaseParser<I, O, R> 

val BaseParser<S, S, R>.many: BaseParser<S, S, List<R>>
val BaseParser<S, S, R>.some: BaseParser<S, S, List<R>>
\end{lstlisting}

Комбинатор \verb|seq| (листинг \ref{lst:combinators}) позволяет объединить два парсера последовательно, так, что выходящие состояния первого парсера передаются во второй.
Результаты же обоих парсеров объединяются в пару.
Также реализованы специальные комбинаторы \verb|seql| и \verb|seqr|, которые действуют как \verb|seq|, но возвращают только левый или правый результат. Это может быть полезно, когда мы хотим проигнорировать какие-то ключевые слова в тексте или определенные вершины в графе. Пример использования таких комбинаторов можно увидеть в листинге \ref{lst:stringParser}: они использовались для игнорирования строкового представления операторов и скобок.

Комбинатор \verb|or| (листинг \ref{lst:combinators}) позволяет объединить два парсера параллельно.
Результирующий парсер будет передавать входящее состояние в оба парсера и объединять их результаты и выходящие состояния.
Поэтому оба парсера должны принимать и возвращать одинаковые типы состояний и результата.

Комбинатор \verb|or| обычно используется чтобы описать несколько парсеров у одного правила в грамматике, но для этого нужно оборачивать каждый парсер в скобки, т.к. в Kotlin нельзя задать приоритет объявленной инфиксной операции. Именно для этого был введена функция \verb|rule| (листинг \ref{lst:combinators}), которая принимает любое количество парсеров как аргументы, и возвращает их параллельное объединение. Пример использования \verb|rule| приведен в листинге \ref{lst:stringParser}.

Следующий комбинатор --- это \verb|using| (листинг \ref{lst:combinators}), которая является альтернативой функции \texttt{map} из библиотеки Trails. Эта функция позволяет заменить результаты парсера на другие с помощью функции \verb|f|.
Для этого он создает новый парсер, который внутри реализует эту логику, сохраняя исходный парсер и функцию \verb|f|.

Но иногда \verb|using| применяется к последовательности комбинаторов.
Например, возьмем такой парсер:

\begin{nobreaks}
    \begin{lstlisting}
(v() seq outE() seq outV()) using { (tmp, u) -> 
    val (v, e) = tmp
    ... 
}
\end{lstlisting}
\end{nobreaks}

Мы знаем, что \verb|seq| создает парсер, который возвращает пару результатов, соответственно парсер выше возвращает \verb|Pair<Pair<V, E>, V>|.
В Kotlin, к сожалению, нет вложенного деструктурирования, чтобы получить доступ к обоим вершинам и ребру за один раз.
Если мы хотим деструктурировать такой результат, нам бы пришлось делать это дважды, как в примере выше, что неудобно.
Для решения это проблемы добавлены перегрузки функции \verb|using|, отличающиеся типом результата у парсера, а именно, когда парсер возвращает вложенные пары. Такие перегрузки принимают функцию \verb|f| с арностью больше один. С ними пример выше можно переписать так:
\begin{lstlisting}
(v() seq outE() seq outV()) using { v, e, u -> ... }
\end{lstlisting}

Другим подходом для решения этой проблемы могли бы быть перегрузки комбинатора \verb|seq|, которые бы объединяли пары результатов в кортежи, которые уже можно деструктурировать за один раз. Проблема тут в том, что в Kotlin нет поддержки кортежей, из-за чего нужно было бы самостоятельно определить типы кортежей разной длины.

В строковых языках запросов часто используется блок \texttt{where} для задания определенных ограничений. Чтобы частично реализовать такое поведение, был введен комбинатор \texttt{that} (листинг \ref{lst:combinators}). Его сигнатура похожа на комбинатор \texttt{seq}: он также принимает два парсера, где второй применяется к результатам первого. Но возвращается парсер, который действует также, как и первый, но результаты которого отфильтрованы. В роли предиката выступает второй парсер: результат отбрасывается, если второй парсер на нем не вернул ни одного результата. Второй парсер также можно назвать подзапросом.


\begin{lstlisting}[float=ht, label=lst:marysLover, caption={Пример запроса к графу с комбинатором \texttt{that}}]
val person = v()
val isMary = vPred { v -> v.value == "Mary" }
val isLoves = ePred { e -> e.label == "loves" }
val marysLover = person
            .that(outE(isLoves) seq outV(isMary))
            .that(inE(isLoves) seq inV(isMary))
val friend = outE { it.label == "friend" }
val marysLoversFriend = marysLover seqr friend seqr outV()
\end{lstlisting}

Пример использования комбинатора \texttt{that} приведен в листинге \ref{lst:marysLover}. В нем мы находим всех людей, которые любят Мэри и Мэри любит их в ответ. После чего для всех таких людей мы находим их друзей. Написать такую логику без комбинатора \texttt{that} было бы сложно, так как другие комбинаторы меняют состояние исходного парсера, а комбинатор \texttt{that} --- нет.

Чтобы писать парсеры циклов, мы можем использовать комбинатор \texttt{many} (листинг \ref{lst:combinators}).
В качестве аргумента он принимает парсер, который имеет одинаковый тип входящего и выходящего состояния.
Этот комбинатор создает парсер, который действует как переданный, но последовательно объединенный сам с собой ноль и более раз.
Созданный парсер возвращает списки результатов длины \texttt{k}, где \texttt{k} --- количество последовательных повторений исходного парсера.
Также написан аналогичный комбинатор \texttt{some}, но для повторений от одного и более раз.

Следующей проблемой является описание рекурсивных парсеров. В Kotlin нельзя использовать переменную в ее же инициализации. А для взаимно-рекурсивных определений мы не можем использовать переменную, которая будет объявлена дальше по коду.

Подход, который предлагается в работе \cite{Meerkat}, заключается в использовании комбинатора неподвижной точки.
Он позволяет инициализировать парсер, используя в определении этот же парсер. Вот пример использования:
\begin{lstlisting}
val s = fix { s -> ("a".p seql s).many }
\end{lstlisting}
Комбинатор \texttt{fix} принимает функцию, которая возвращает парсер, инициализированный с использованием  аргумента \texttt{s}. Этот аргумент является тем самым парсером, который эта функция создает.
Тип комбинатора \texttt{fix} выглядит так:

\begin{nobreaks}
    \begin{lstlisting}
fun fix(f: (BaseParser<I, O, R>) -> BaseParser<I, O, R>)
    : BaseParser<I, O, R>
\end{lstlisting}
\end{nobreaks}

Использование комбинатора \texttt{fix} имеет свои недостатки: нужно дублировать имя переменной и нельзя объявить взаимно-рекурсивные парсеры. Для взаимной рекурсии можно создать другие определения \texttt{fix}, которые будут принимать функции с арностью больше один. Тогда такая функция должна возвращать столько же парсеров, сколько и аргументов она получает, при этом каждое определение парсера соответствует своему аргументу. А учитывая, что в Kotlin нет удобной поддержки кортежей, такой подход будет неудобным.

Для решения проблемы с взаимно-рекурсивным определением, я добавил парсеры с отложенной инициализацией, названные \texttt{LazyParser}. Использование такого парсера можно увидеть в листинге \ref{lst:stringParser}. Такой подход больше соответствует императивному стилю программирования, что может быть удобным для некоторых пользователей. Также с помощью таких парсеров мы можем определить взаимно-рекурсивные парсеры, например, сначала объявив нужные парсеры, а затем инициализировав их. Но у данного подхода есть проблема: проверка того, что парсер был инициализирован, происходит только во время исполнения, а не во время компиляции. У комбинатора \texttt{fix} такой проблемы нет: мы не можем создать парсер, не инициализировав его. К сожалению, в случае \texttt{LazyParser}, мы не можем с помощью системы типов определить, был ли парсер инициализирован.





\section{Мемоизация состояний и результатов}

Как упоминалось ранее, комбинаторы, реализованные через рекурсивный спуск, не способны поддерживать левую рекурсию в определении грамматики, а также циклы в графе.

\begin{lstlisting}[float=ht, label=lst:leftRec, caption=Пример левой рекурсии в грамматике]
    val s = fix { s -> s seq outE() seq outV() or eps() }
\end{lstlisting}

В качестве примера возьмем код с листинга \ref{lst:leftRec}. Допустим, в комбинаторе \texttt{or} мы сначала вызываем первый парсер. Комбинатор \texttt{seq} определен так, что он сначала вызывает первый парсер и на каждом его результате вызывает второй. Таким образом, при вызове парсера \texttt{s}, мы рекурсивно спускаемся до этого же парсера \texttt{s}, и этот процесс будет повторятся бесконечно.

Для решения проблемы, описанной выше, мы хотим детектировать случаи, когда мы рекурсивно спустились в тот же парсер. Для этого добавим мемоизацию на уровне парсера.

Давайте для каждого входящего состояния запоминать результат, который мы возвращали для него ранее. Тогда при повторении входящего состояния для парсера, мы не будем снова вычислять результат, а вернем тот, который запомнили ранее.

Но какой результат возвращать при первом вызове парсера \texttt{s} в листинге \ref{lst:leftRec}?
Его сначала нужно вычислить, но это приведет к рекурсивному вызову.
Тут можно воспользоваться идеей из работы \cite{MemoizationJohnson}, где предлагается использовать стиль передачи продолжений (Continuation-Passing Style, CPS) при определении парсера.

Обычная функция в привычном понимании при вызове с некоторыми аргументами возвращает результат. Функция \texttt{f}, написанная в CPS стиле, вместе c аргументами принимает продолжение (continuation) и не возвращает результат. Продолжение само по себе является функцией, которая принимает результат функции \texttt{f}. Функция \texttt{f} при вычислении передает свои результаты в продолжение. Продолжение представляет из себя некоторое вычисление, которое как-то обрабатывает каждый результат функции \text{f}.

\begin{lstlisting}[float=ht, label={lst:CpsExample}, caption=Пример функции в CPS стиле]
fun foo(x: A, y: B): List<R> {
    return listOf(x.bar(y), x.baz())
}

fun fooCPS(x: A, y: B, c: (R) -> Unit): Unit {
    c(x.bar(y))
    c(x.baz())
}
\end{lstlisting}

Например, в листинге \ref{lst:CpsExample} функция \texttt{foo} написана в обычном стиле, и она возвращает несколько результатов как список. Такую функцию можно переписать, используя CPS стиль, получив в итоге функцию \texttt{fooCPS}, которая каждый результат передает в продолжение \texttt{c}.

Как было сказано ранее, парсер является функцией, которая принимает входящее состояние, производит некоторое вычисление и возвращает множество пар выходящих состояний и результатов. Мы хотим для каждого входящего состояния запоминать результат работы парсера. Пусть тогда парсер вместо непосредственно результатов возвращает некоторое вычисление и запоминает его. Это вычисление будет является функцией, написанной в стиле передачи продолжений. 

Ранее парсер возвращал \lstinline|Sequence<ParserResult>|, где \lstinline|ParserResult| содержал результат и выходящее состояние. Теперь парсер будет возвращать только \lstinline|ParserResult|, который будет представлять вычисление. Тогда оно будет определено так:

\begin{lstlisting}[float=ht]
typealias Continuation<S, R> = (S, R) -> Unit
class ParserResult<S, R>(
        val f: (Continuation<S, R>) -> Unit) {...}
\end{lstlisting}

Это вычисление будет представлять из себя работу парсера. При этом каждые вычисленные результат и выходящее состояние будут передаваться в продолжение.

Теперь парсер может для каждого входящего состояния запоминать вычисление, которое он вернул, чтобы возвращать ту же функцию в следующий раз. Теперь нам нужно обеспечить, чтобы это вычисление исполнялось один раз.

\begin{lstlisting}[language=Matlab, label={lst:memoResult}, float=ht, caption=Мемоизация результата парсера]
function memoResult(parserRes):
results = empty list of pairs (S, R)
continuations = empty list of Continuation<S, R>

return function (k):
    isFirstCall = continuations is empty
    add k to continuations
    
    if isFirstCall:
        parserRes.apply(function (s, r):
            newResult = (s, r)
            if newResult not in results:
                add newResult to results
                for each continuation in continuations:
                    continuation(s, r))   
    else:
        for each (s, r) in results:
            k(s, r)
\end{lstlisting}

Для этого обернем вычисление, которое возвращает парсер, в другое вычисление, которое будет обеспечивать единственность выполнения.
Псевдокод функции \texttt{memoResult}, которая это делает, представлен в листинге \ref{lst:memoResult}.

Будем для вычисления парсера хранить список результатов, которые он вернул, и список продолжений, с которыми это вычисление вызывали. Когда вычисление парсера впервые вызывают, список с продолжениями пуст. Тогда мы и будем впервые исполнять исходное вычисление. Как известно, оно принимает продолжение, поэтому мы создадим свое продолжение и передадим его. В нем мы будем получать выходящие состояния и результаты парсера на текущем входящем состоянии. Если такой результат мы не получали ранее, то передадим его всем продолжениям, которые мы запомнили.

Если вычисление вызывается повторно, то мы не будем исполнять исходное вычисление парсера, а просто передадим в полученное продолжение все известные на данный момент результаты, при этом сохранив это продолжение на случай, если в будущем появятся новые результаты.

Рассмотрим выполнение парсера \texttt{s} в листинге \ref{lst:leftRec} с такой реализацией парсера.
Когда мы впервые вызываем парсер \texttt{s}, мы передаем в него продолжение, которое будет принимать его результаты.
Это продолжение мы запоминаем, и вызываем вычисление парсера \texttt{s}.
Там мы рекурсивно спустимся в левую ветку \texttt{or}, пройдем через пару \texttt{seq} и снова вызовем парсер \texttt{s}.
На этот раз мы будем вызывать его с продолжением, в котором будет вызываться следующий за ним парсер \texttt{outE()} в последовательности \texttt{s seq outE() seq outV()}.
На этот момент парсер \texttt{s} не запомнил ни одного результата, поэтому переданное продолжение вызвано не будет, но оно будет запомнено.
Далее мы вернемся на уровень комбинатора \texttt{or}, и вызовем правый парсер.
Там находится комбинатор \texttt{eps}, который всегда возвращает результат \texttt{Unit}.
Этот \texttt{Unit} будет возвращен как результат парсера \texttt{s}.
Получим этот результат мы в вычислении, которое сохранили в функции \texttt{memoResult}, где на данный момент запомнили два продолжения.
Когда этот результат будет передаваться в первое продолжение, его получит пользователь парсера \texttt{s}.
Когда же он будет передаваться во второе продолжение, его получит парсер \texttt{outE()}, следующий за парсером \texttt{s}.
Если в графе есть путь с исходящим ребром и вершиной, то парсер \texttt{s} вернет еще один результат, который также будет передан в два сохраненных продолжения.
Это будет продолжаться до тех пор, пока в графе есть подходящие грамматике пути.

Так мы смогли поддержать леворекурсивные грамматики.
А также такая реализация парсера, описанная в работе \cite{MemoizationJohnson}, улучшает скорость работы разбора с экспоненциальной до полиномиальной.
Но что, если в графе будет цикл?
Тогда путей, соответствующих грамматике, будет бесконечно много, и вычисление парсера \texttt{s} будет бесконечным.

\section{Сжатый лес разбора}

Для решения проблемы с бесконечным количеством результатов парсер должен возвращать некоторую конечную структуру данных, которая может представлять все возможные разборы парсера.

\begin{figure}[t]
    \centering
    \begin{subfigure}[b]{0.25\textwidth}
        \centering
        \includesvg[inkscapelatex=false,width=\textwidth]{sppf/gr.svg}
        \caption{Граф}
        \label{fig:subGraph}
    \end{subfigure}
    \hfill
    \begin{subfigure}[b]{0.55\textwidth}
        \centering
        \includesvg[inkscapelatex=false,width=\textwidth]{sppf/1.svg}
        \caption{Первое дерево разбора}
        \label{fig:subDerivedTrees1}
    \end{subfigure}
    \vskip\baselineskip
    \begin{subfigure}[b]{1.00\textwidth}
        \centering
        \includesvg[inkscapelatex=false,width=\textwidth]{sppf/2.svg}
        \caption{Второе дерево разбора}
        \label{fig:subDerivedTrees2}
    \end{subfigure}


    \caption{Пример деревьев разбора}
    \label{fig:derivedTrees}
\end{figure}

\begin{figure}[t]
    \centering
    \begin{subfigure}[b]{0.8\textwidth}
        \centering
        \includesvg[inkscapelatex=false,width=\textwidth]{sppf/3.svg}
        \caption{SPPF}
        \label{fig:subSppf}
    \end{subfigure}
    \hfill
    \begin{subfigure}[b]{0.7\textwidth}
        \centering
        \includesvg[inkscapelatex=false,width=\textwidth]{sppf/4.svg}
        \caption{Бинаризованный SPPF}
        \label{fig:subBinSppf}
    \end{subfigure}

    \caption{Пример обычного и бинаризованного SPPF}
    \label{fig:sppf}
\end{figure}


Такой структурой может являться сжатое представление леса разбора (Shared Packed Parse Forest, SPPF). Впервые такая структура была использована в работе \cite{RekersSppf}.

Рассмотрим пример разбора графа, изображенного на рисунке \ref{fig:subGraph}. Разбор проводился с помощью парсера \texttt{S} в листинге \ref{fig:sppfParser}.

\begin{lstlisting}[float=h, label=fig:sppfParser, caption=Пример парсера для разбора графа]
val x = outV { it.value == "x" }
val y = outV { it.value == "y" }
val a = outE { it.label == "A" }
val b = outE { it.label == "B" }
val P = a seq y
val S = P or (P seq b seq y)
\end{lstlisting}

Парсер запускался, начиная с вершины \texttt{x}. На рисунке \ref{fig:subDerivedTrees1} и \ref{fig:subDerivedTrees2} изображены два дерева разбора. В них листья соответствуют базовым комбинаторам, а остальные --- объединяющим комбинаторам. Вершина в дереве имеет вид $(p, i, o)$, где $p$ --- парсер или результат, $i$ --- входящее состояние, $o$ --- выходящее состояние. Первое дерево соответствует пути $x \xrightarrow{A} y$, а второе --- пути $x \xrightarrow{A} y \xrightarrow{B} y$.

Таких деревьев разбора может быть бесконечно много, но мы можем объединить их в SPPF. Пример SPPF для упомянутых ранее деревьев разбора приведен на рисунке \ref{fig:subSppf}. В этом представлении появляется новый тип сжатых вершин, на рисунке изображенный небольшим квадратом. Такая вершина представляет из себя один из вариантов разбора родительской вершины. На рисунке \ref{fig:subSppf} видим, что некоторые вершины присутствуют в обоих разборах, например, вершина \texttt{E("A")}.

В работе \cite{BinSppf} было представлено бинаризованное SPPF, что является улучшением обычного SPPF. В бинаризованном виде объем леса достигается $O(n^3)$, где $n$ --- размер входа. В таком SPPF сжатые узлы имеют не более двух детей. Пример сжатого леса разбора приведен на рисунке \ref{fig:subBinSppf}.

Как мы можем заметить, узел SPPF параметризуется парсером, входящим и выходящим состоянием. Даже если у нас будет бесконечное количество деревьев разбора, у нас будет конечный объем SPPF, т.к. количество парсеров и состояний конечно.

\begin{figure}[htp]
    \centering
    \begin{subfigure}[b]{0.3\textwidth}
        \centering
        \includesvg[inkscapelatex=false,width=0.5\textwidth]{sppf/loop_gr.svg}
        \caption{Граф}
        \label{fig:subLoopGr}
    \end{subfigure}
    \hfill
    \begin{subfigure}[b]{0.5\textwidth}
        \centering
        \includesvg[inkscapelatex=false,width=\textwidth]{sppf/loop.svg}
        \caption{SPPF}
        \label{fig:subLoopSppf}
    \end{subfigure}

    \caption{Пример SPPF с циклом}
    \label{fig:sppf}
\end{figure}

Примером случая, когда разбор возвращает бесконечно много результатов, может быть разбора цикла в графе. Возьмем грамматику в листинге \ref{lst:loopParser}.

\begin{lstlisting}[float=h, label={lst:loopParser}, caption=Пример парсера графа с циклом]
val x by outV { it.value == "x" }
val a by outE { it.label == "A" }
val S by LazyParser<SimpleVertexState, SimpleVertexState, Any>()
S.p = (S seq a seq x) or eps()
\end{lstlisting}

Такая грамматика разбирает бесконечно много путей вида $(x \xrightarrow{A} x)*$ в графе на рисунке \ref{fig:subLoopGr}. Лес разбора для такого парсера и графа изображен на рисунке \ref{fig:subLoopSppf}. Как мы можем увидеть, дочерний узел \texttt{(S a)} ссылается на предка \texttt{S}, формируя цикл в SPPF. Такой лес разбора хранит бесконечное количество путей.

\subsection{Генерация сжатого леса разбора}

Чтобы генерировать SPPF как результат разбора, мы должны в вычислении парсера вместо пары <<результат и выходящее состояние>> возвращать вершину SPPF. Рассмотрим, какие вершины будут возвращать комбинаторы.

Все вершины, которые не являются сжатыми, параметризуются начальным и конечным состоянием. Тип несжатой вершины также параметризуется типом результата, который соответствующий ей парсер генерирует:
\begin{lstlisting}
NonPackedNode<LS, RS, R>(val leftState: LS, val rightState: RS)
\end{lstlisting}

Сжатые же вершины хранят ссылку на одного или двух детей, которые в свою очередь являются несжатыми вершинами:

\begin{nobreaks}
    \begin{lstlisting}
class PackedNode<LS, MS, RS, R1, R2>(
    val leftNode: NonPackedNode<LS, MS, R1>?,
    val rightNode: NonPackedNode<MS, RS, R2>
    ...
)
\end{lstlisting}
\end{nobreaks}

Базовые комбинаторы, такие как комбинатор вершины или ребра, генерируют терминальную вершину, которая хранит непосредственно результат: 

\begin{nobreaks}
    \begin{lstlisting}
class TerminalNode<...>(val result: T, val action: (T) -> R, ..)  
    : NonPackedNode<...>(leftState, rightState)
\end{lstlisting}
\end{nobreaks}

Объединяющие комбинаторы генерируют промежуточную вершину. Такая вершина хранит соответствующий парсер, а также список детей, которые являются сжатыми вершинами:

\begin{nobreaks}
\begin{lstlisting}
class IntermediateNode<...>(
    val parser: BaseParser<LS, RS, *>, 
    val action: (Pair<CR1?, CR2>) -> R, ...
) : NonPackedNode<...>(leftState, rightState) {
    val packedNodes: MutableList<PackedNode<..., CR1?, CR2>>
...}
\end{lstlisting}
\end{nobreaks}

Например, комбинатор \texttt{seq} получает две вершины от парсеров и создает промежуточную вершину, к которой через сжатую вершину подвешивает эти две вершины. 

Терминальная и промежуточная вершина также хранят семантическое действие, которое заменяет исходный результат на другой. Такое действие добавляется при использовании комбинатора \texttt{using}. Оно будет использоваться при извлечении результатов из SPPF.

Помимо непосредственной генерации вершин SPPF нам нужно обеспечить раздельное использование одинаковых вершин между парсерами. Упомянутый ранее комбинатор \texttt{seq} должен либо создавать новую промежуточную вершину, либо использовать созданную ранее, чтобы подвесить к ней сжатую вершину. Для этого заведем хранилище вершин, которое будет обеспечивать раздельное доступ к вершинам. Такое хранилище будет представлять из себя множество вершин, которое позволяет добавлять новые вершины в это множество, если таких еще нет, или доставать из него уже созданные ранее вершины. 

Данное хранилище будет передаваться парсеру на вход, а парсер в свою очередь будет передавать его во вложенные парсеры рекурсивно, таким образом все парсеры будут иметь доступ только к одному хранилищу, а значит будут разделять одни и те же вершины. 

\subsection{Извлечение результатов из леса разбора}

Теперь парсер в качестве результата возвращает некоторое количество вершин SPPF, причем конечное. Каждая вершина представляет из себя множество деревьев разбора, причем неограниченного размера.

Для извлечения результатов из SPPF, мы можем совершить его обход, например, в ширину. Такой обход позволяет поддержать возможные циклы в SPPF, но написать извлечение результатов с обходом в ширину не тривиально: нужно как-то собирать результаты с терминальных вершин, и при этом объединить их в промежуточных и сжатых вершинах. 

Для простоты реализации был написан рекурсивный обход в глубину, реализованный с помощью структуры данных \texttt{Sequence} из стандартной библиотеки Kotlin, представляющую ленивую последовательность данных. Каждый тип вершины умеет возвращать \texttt{Sequence} результатов. Промежуточная вершина объединяет последовательности результатов сжатых вершин в одну. Терминальная вершина возвращает последовательность из одного элемента, который она хранит. Сжатая вершина возвращает последовательность пар результатов, где каждый результат первого ребенка объединяется с каждым результатом второго ребенка, если второй ребенок есть. 

При этом промежуточная и терминальная вершина перед возвратом каждого результата применяет к нему семантическое действие, которое хранится в вершине. 

Такой обход в глубину был протестирован на графах с циклами, а также на леворекурсивных грамматиках. В итоге не удалось написать такой тест, в котором бы обход в глубину зациклился и не завершился. Это можно объяснить тем, что если в SPPF есть цикл, то обход в глубину на каждой итерации этого цикла будет добавлять ненулевой результат в последовательность. Обход в глубину перебирает сжатые вершины у промежуточной вершины в порядке их добавления. А благодаря мемоизации результатов парсера, первой будет добавляться вершина, которая соответствует той ветви парсера, где нет рекурсивного вызова этого же парсера. Соответственно, из первой сжатой вершины нельзя прийти в цикл. А значит она будет возвращать ненулевой результат, который будет добавлен в последовательность.

